\documentclass[psamsfonts]{amsart}

\usepackage{bussproofs}

%% Macros
\def\G{\Gamma}
\def\D{\Delta}
\def\th{\vdash}
\def\imp{\Rightarrow}

\def\right#1{$\mathrm{#1}_\mathcal{R}$}
\def\left#1{$\mathrm{#1}_\mathcal{L}$}

\def\fsp{\textsf{Fellowship}}
\def\lbmm{\bar{\lambda}\mu\tilde{\mu}}

%% Title 
\title[Fellowship]{Fellowship: Who needs a manual anyway?}
\author[F. Kirchner]{Florent Kirchner}
%\author{}   add yourself if you dare!
\address{LIX, \'Ecole Polytechnique, 91128 Palaiseau, France}


\begin{document}

\begin{abstract}
Indeed, the \texttt{help} command of the \fsp\ toplevel is your best friend.
However, a deeper insight of the logical structure of the prover might prove
helpful, thus this note.
\end{abstract}
\maketitle


%%%%%%%%%%%%%%%%%%%%%%%%%%%%%%%%%%%%%%%%%%%%%%%%%%%%%%%%%%%%%%%%%%%%%%%%%%%%%%%
\section{Inference rules --\textsl{a.k.a. typing rules, a.k.a. tactics}}

\subsection{Syntax}

$\lbmm$-terms~\cite{Lbmm:CurienHerbelin00} are used as proof terms. The syntax
of the $\lbmm$-calculus defines commands $c$, terms $v$ and contexts $e$ as
follows:
\begin{eqnarray*}
c &::=& \langle v|e \rangle \\
v &::=& x\ \|\ \mu\alpha.c\ \|\ \lambda x.v\ \|\ e\cdot v \\
e &::=& \alpha\ \|\ \tilde{\mu}x.c\ \|\ v\cdot e\ \|\ \alpha\lambda'.e
\end{eqnarray*}

\subsection{Classical logic sequent calculus.}

Figure~\ref{LK} describes the inference rules.

\begin{figure}[htbp]
\fbox{
\begin{minipage}{\textwidth}

% AXIOM
\begin{center}
\AxiomC{}
\LeftLabel{\left{axiom}}
\UnaryInfC{$ \G ; *:A \th \D,x:A,\D' $}
\DisplayProof
\quad
\AxiomC{}
\LeftLabel{\right{axiom}}
\UnaryInfC{$\G,x:A,\G' \th *:A ; \D $}
\DisplayProof
\end{center}

% TRUE - FALSE
\begin{center}
\AxiomC{}
\LeftLabel{\left{false}}
\UnaryInfC{$ \G ; *:\textsl{false} \th \D $}
\DisplayProof
\quad
\AxiomC{}
\LeftLabel{\right{true}}
\UnaryInfC{$\G \th *:\textsl{true} ; \D $}
\DisplayProof
\end{center}

% CUT
\begin{center}
\AxiomC{$x:A , \G \th *:B ; \D$}
\AxiomC{$x:a , \G ; *:B \th \D$}
\LeftLabel{\left{cut}}
\BinaryInfC{$\G ; *:A \th \D$}
\DisplayProof
\quad
\AxiomC{$\G \th *:B ; x:A, \D$}
\AxiomC{$\G ; *:B \th x:A, \D$}
\LeftLabel{\right{cut}}
\BinaryInfC{$\G \th *:A ; \D$}
\DisplayProof
\end{center}

% IMPLY
\begin{center}
\AxiomC{$\G \th *:A ; \D$}
\AxiomC{$\G ; *:B \th \D$}
\LeftLabel{\left{imply}}
\BinaryInfC{$\G ; *:A\imp B \th \D$}
\DisplayProof
\quad
\AxiomC{$\G, x:A \th *:B, \D$}
\LeftLabel{\right{imply}}
\UnaryInfC{$\G \th *:A\imp B ; \D$}
\DisplayProof
\end{center}

% MINUS
%\begin{center}
%\AxiomC{$\G; *:A \th x:B, \D$}
%\LeftLabel{\left{minus}}
%\UnaryInfC{$\G ; *:A - B \th \D$}
%\DisplayProof
%\quad
%\AxiomC{$\G ; *:B \th \D$}
%\AxiomC{$\G \th *:A; \D$}
%\LeftLabel{\right{minus}}
%\BinaryInfC{$\G \th *: A - B ; \D$}
%\DisplayProof
%\end{center}

% CONJ (and - wedge)
\begin{center}
\AxiomC{$\G, x:A, y:B \th *:C ; \D$}
\AxiomC{$\G, x:A, y:B ; *:C \th \D$}
\LeftLabel{\left{conj}}
\BinaryInfC{$\G ; *:A\wedge B$}
\DisplayProof
\quad
\AxiomC{$\G \th *:A ; \D$}
\AxiomC{$\G \th *:B ; \D$}
\LeftLabel{\right{conj}}
\BinaryInfC{$\G \th *:A\wedge B ; \D$}
\DisplayProof
\end{center}

% DISJ (or - vee)
\begin{center}
\AxiomC{$\G ; *:A \th \D$}
\AxiomC{$\G ; *:B \th \D$}
\LeftLabel{\left{disj}}
\BinaryInfC{$\G ; *:A\vee B \th \D$}
\DisplayProof
\end{center}
\begin{center}
\AxiomC{$\G \th *:A ; \D$}
\LeftLabel{\right{disj_1}}
\UnaryInfC{$\G \th *:A\vee B ; \D$}
\DisplayProof
\quad
\AxiomC{$\G \th *:B ; \D$}
\LeftLabel{\right{disj_2}}
\UnaryInfC{$\G \th *:A\vee B ; \D$}
\DisplayProof
\end{center}

% NEG
\begin{center}
\AxiomC{$\G \th *:A ; \D$}
\LeftLabel{\left{neg}}
\UnaryInfC{$\G ; *:\neg A \th \D$}
\DisplayProof
\quad
\AxiomC{$\G ; *:A \th \D$}
\LeftLabel{\right{neg}}
\UnaryInfC{$\G \th *:\neg A ; \D$}
\DisplayProof
\end{center}

% FORALL
\begin{center}
\AxiomC{$\G ; *:B[x\leftarrow t] \th \D$}
\LeftLabel{\left{forall}}
\UnaryInfC{$\G ; *:\forall (x:A) B \th \D$}
\DisplayProof
\quad
\AxiomC{$\G \th *:B ; \D$}
\LeftLabel{\right{forall}}
\UnaryInfC{$\G \th *:\forall (x:A) B ; \D$}
\DisplayProof
\end{center}

% EXISTS
\begin{center}
\AxiomC{$\G ; *:B \th \D$}
\LeftLabel{\left{exists}}
\UnaryInfC{$\G ; *:\exists (x:A) B \th \D$}
\DisplayProof
\quad
\AxiomC{$\G \th *:B[x\leftarrow t] ; \D$}
\LeftLabel{\right{exists}}
\UnaryInfC{$\G \th *:\exists (x:A) B ; \D$}
\DisplayProof
\end{center}

\end{minipage}
}
\caption{Classical logic sequent calculus.}
\label{LK}
\end{figure}

In the case of substractive logic, the elimination rules for the $-$
connective are added:
% MINUS
\begin{center}
\AxiomC{$\G; *:A \th x:B, \D$}
\LeftLabel{\left{minus}}
\UnaryInfC{$\G ; *:A - B \th \D$}
\DisplayProof
\quad
\AxiomC{$\G ; *:B \th \D$}
\AxiomC{$\G \th *:A; \D$}
\LeftLabel{\right{minus}}
\BinaryInfC{$\G \th *: A - B ; \D$}
\DisplayProof
\end{center}

Finally in the case of minimal logic, the \left{neg} and \right{neg} rules are
discarded, as well as the \left{false} rule. $\neg A$ is considered as
syntactic sugar for $A \imp \textsl{false}$, and is eliminated using the rules
for implication.

%%%%%%%%%%%%%%%%%%%%%%%%%%%%%%%%%%%%%%%%%%%%%%%%%%%%%%%%%%%%%%%%%%%%%%%%%%%%%%%
\bibliographystyle{amsplain} 
\bibliography{bib}

\end{document}
% That's all folks !


\AxiomC{$$}
\AxiomC{$$}
\LeftLabel{}
\BinaryInfC{$$}
\DisplayProof

